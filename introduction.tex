\section{Introduction}
\label{sec:This}

One of the advantages often attributed to purely functional programming languages is that they are great for \gls{verification} and \emph{equational reasoning} \cite{Wadler87}.
When I started learning to program in a functional language, I didn't know precisely what reasoning about program is and how it can be used to verify certain \emph{properties} of a program. I also didn't know what kind of properties of a program can be verified with formal methods. In order to understand equational reasoning I wrote this article.

\subsection{Why use equational reasoning?}

Equational reasoning can be used as a formal method for software verification. It's a way to prove the correctness of a program. Correctness means that specified properties of the program hold for an arbitrary input. 

A property of a program can be specified in the form of an equation. For example, the following equation states a property of the function \verb|fmap|  \footnote{The function \verb|fmap| takes a function \verb|a -> b| as its first parameter and a container that contains elements of type \verb|a|. It returns a container with elements of type \verb|b|. This is called \emph{mapping}. The type of \verb|fmap| is \verb|(a -> b) -> f a -> f b|.}.
\begin{equation}
  \label{eq:firstfunctorlaw}
\text{fmap } \text{id}  =  \text{id}  
\end{equation}
 The function \verb|id| is the identity function. The property represented by equation \ref{eq:firstfunctorlaw} says that applying the function \verb|fmap| with the arguments \verb|id| and an arbitrary container is the same as applying the function \verb|id| on a container. Simply put, mapping the identity function over every item in a container has no effect. Given a function definition of \verb|fmap|, we can check if this property holds, using equational reasoning.

The goal of equational reasoning is to  provide certainty that a program has particular properties. Proof of required properties contributes to the reliability and robustness of a software. 
For example, the first formally verified micro kernel, seL4, was verified using Haskell and formal verification to ensure proper security functionality and security assurance \cite{Klein09}.
Another example is the streaming library "pipes". The author, Gabriel Gonzales, used equational reasoning to prove the correctness of the library \cite{gonzales13}.

\subsection{Why are functional programming languages great for equational reasoning?}

An equation is a statement that signifies the equality of two expressions (e.g. $1 + 3 = 4$). Reasoning is the process of thinking about something in a logical way in order to form a conclusion. The technique of equational reasoning implies that we show correctness by showing that two expressions of a program are equal. 

The reason we can use equational reasoning in functional programming is \gls{referential_transparency}. A piece of code can be replaced by its value; this is true in purely functional languages. The absence of side effects allows us to replace the left-hand side of a definition with the right-hand side and vice versa. For example if you have the following Haskell definition:
\begin{verbatim}
x = 23
\end{verbatim}
That means that any \verb|x| can be substituted by \verb|23|.

This is not possible in programs written in an imperative programming language because there the value of an expression depends on the context of the execution environment. In an imperative language, \verb|x = 23| is an assignment, not an equality. The variable \verb|x| is mutable and the value of the expression \verb|x| can change at runtime. Hence, we must not substitute \verb|x| with \verb|23| in your program because arbitrary values could be assigned to the variable \verb|x| in other statements.

Haskell is a purely functional language, which means that functions do not have side effects. Hence, we can verify programs written in Haskell using equational reasoning. For this reason, all examples in this article are written in Haskell.

\subsection{What properties will we discuss?}

If we want to verify a program, we have to define one or more properties. These properties are part of the program specification. We could define arbitrary properties, but this article mainly deals with properties derived from the so called \emph{type class laws}. In Haskell, type class laws are exhibited by \glspl{typeclass}. For example, the property formed by equation \ref{eq:firstfunctorlaw} is a type class law of the type class \verb|Functor| (cf. appendix \ref{sec:functor}).

Type classes are a concept to describe the behavior of a type. 
They are very similar to interfaces. If we want to make a type an instance of a type class, we have to implement the functions given by the \glspl{typesignature} of the type class. Figure \ref{fig:typeclassrelation} illustrates the relation between values, types and type classes in Haskell. A type can be a member of several type classes. A type class can contain several types. A value has exactly one type but several values can be of the same type.

An instance of a type class must obey the corresponding type class laws. These laws allow us to make assumptions about the behavior of the program and to reason about the code. Properties of existing code allow us to rely on expected behavior and deduce further properties for new code. For example, if the property formed by equation \ref{eq:firstfunctorlaw} holds, we can rest assured that \verb|fmap| only applies the function to the items in the container and has no additional effects. If an implementation of \verb|fmap| would change the structure of the container, it would break the property of equation \ref{eq:firstfunctorlaw}.

The Haskell compiler does not enforce type class laws. We need to verify them ourselfes when implementing an instance of a type class. We can use equational reasoning to prove that an implementation of a type class obeys the corresponding laws.

\tikzset{multi  attribute/.style={attribute ,double  distance=1.5pt}}
\tikzset{derived  attribute/.style={attribute ,dashed}}
\tikzset{total/.style={double  distance=1.5pt}}
\tikzset{every  entity/.style={draw=orange , fill=orange!20}}
\tikzset{every  attribute/.style={draw=MediumPurple1, fill=MediumPurple1!20}}
\tikzset{every  relationship/.style={draw=Chartreuse2, fill=Chartreuse2!20}}

\begin{figure}
\centering
\begin{tikzpicture}[node distance=10em]
  \node[entity](type){type};
  \node[entity](typeclass)[right of=type]{type class};
  \node[entity](value)[left of=type]{value};
  \draw (type) -- (typeclass) node [very near end, above=1pt] {*} node [very near start, above=1pt] {*};
\draw (value) -- (type) node [very near end, above=1pt] {1} node [very near start, above=1pt] {*};
  
\end{tikzpicture}
\caption{Relation between value, type and type class}
\label{fig:typeclassrelation}
\end{figure}
\subsection{Overview}

This article will illustrate in detail how equational reasoning works in practice using the type class laws of the type class \verb|Monoid| as the running example. The article doesn't require prior knowledge of type classes or \glspl{monoid}. In order to understand every step of the process, the next two sections will explain the fundamental concepts.
Section \ref{sec:typeclasses} will give a short introduction to type classes and section \ref{sec:monoid} describes monoids. In section \ref{sec:equationalreasoning} we will give an introduction to equational reasoning in general. In section \ref{sec:example} we will prove that an implementation of the \verb|Monoid| type class obeys the first monoid law.  


This article give answers to the following questions:
\begin{itemize}
\item What are desired properties of a program? Where do they come from? Why are they useful?
\item What is equational reasoning? What is the difference between testing and property proving? How do we use equational reasoning to verify properties of a program?
\end{itemize}

