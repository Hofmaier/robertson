\section{Introduction}
\label{sec:intro}

Type classes are a concept to describe behavior. In addition, they exhibit certain properties. This properties are called laws. 
Laws allow us to make assumption about the behavior of the program and reason about the code. Properties of existing code allow us to rely on expected behavior and deduce further properties for new code.

A compiler doesn't verify if a given property holds. It's in the responsibility of the programmer to check if the program behaves correctly. For most programmers it's hard to write a correct program at the first attempt. This leaves us with the question of how to verify the type class laws. There are several methods to verify software. In this article we will look at technique of equational reasoning to prove type class laws. 

Equational reasoning is a formal method for software verification with the aim of providing a high reliability. 
Correctness is specified in form of equations. The source code must satisfy this equations. 

For example, the first formally verified micro kernel, seL4 was verified using formal verification. The programming language Haskell was used because it's easier to reason about a program written in a purely functional programming language, like Haskell, compared to a program written in a imperative or object-oriented language  \cite{Klein09}. 

In this article we first describe type classes and the corresponding laws with a few examples. Then we show that equational reasoning can be used to verify properties of a haskell program. The described methods will be demonstrated with a simple example. We will prove that the monoid laws holds for new type using polymorphism and equational reasoning. The last section we show that equational reasoning is used in a common library like pipes. All examples in this article a written in Haskell.

