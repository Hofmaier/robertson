\section{Introduction}
\label{sec:This}

One of the advantages often claimed by pure functional programming languages is that they are great for \gls{verification} and \emph{equational reasoning} \cite{Wadler87}.
When I started learning to program in a functional language I didn't know precisely what equational reasoning is and how it can be used to verify certain \emph{properties} of a program. I also didn't know what kind of properties of a program can be verified with equational reasoning. In order to grasp equational reasoning I wrote this article.
This article describes the verification method equational reasoning by an intermediate example. 

\subsection{Why use equational reasoning?}

Equational reasoning can be used as a formal method for software verification. It's a way to proof the correctness of a program. Correctness means that specified properties of the program hold for an arbitrary input. 

A property of a program can be specified in the form of an equation. For example
\begin{equation}
  \label{eq:firstfunctorlaw}
\text{fmap } \text{id}  =  \text{id}  
\end{equation}
is a property of the function \verb|fmap|. \footnote{The function \verb|fmap| takes a function \verb|a -> b| as it's first parameter and a container that contains elements of type \verb|a|. It returns a container with elements of type \verb|b|. This is called \emph{mapping}. The type of \verb|fmap| is \verb|(a -> b) -> a -> b|.} The function \verb|id| is the identity function. The property represented by equation \ref{eq:firstfunctorlaw} says that applying the function \verb|fmap| with the argument \verb|id| is the same as applying the function \verb|id|. Put it another way, mapping the identity function over every item in a container has no effect. Given a function definition of \verb|fmap|, we can proof that this property holds using equational reasoning.

The goal of equational reasoning is to provide high reliability. 
For example, the first formally verified micro kernel, seL4 was verified using formal verification to ensure proper security functionality and security assurance \cite{Klein09}.
Another example is the streaming library pipes. The author, Gabriel Gonzales, used equational reasoning to prove the correctness of the library \cite{gonzales13}.

\subsection{Why are functional programming languages great for equational reasoning?}

An equation is a statement that two expressions are equal (e.g. $1 + 3 = 4$). And reasoning is the process of thinking about something in a logical way in order to form a conclusion. The technique of equational reasoning implies that we show correctness by showing that two expression of a program are equal. 

The reason why we can use equational reasoning in functional programming is \gls{referential_transparency}. A piece of code can be replaced by its value; this is true in purely functional languages. The absence of side effects allow us the replace the left-hand side of a definition with the right-hand side and vice verca. For example if you have the following Haskell definition:
\begin{verbatim}
x = 23
\end{verbatim}
That means wherever you see an \verb|x| in your program you can substitute it with \verb|23|.

This isn't possible in programs written in an imperative language because they are dependent on their context. In an imperative language \verb|x = 23| is an assigment, not a definition. If the statement is an assigment we must not substitute \verb|x| with \verb|23| in your program because arbitrary values could be assigned to the variable \verb|x| in other statements of the program.

The programming language Haskell is a purely functional language, which means that in general, functions do not have side effects. Hence, we can verify Haskell program using equational reasoning. For this reason all examples in this article a written in Haskell.

\subsection{What properties will we discuss?}

Equational reasoning is used to proof if certain properties hold. If we want to verify a program, we have to define one or more properties in the first place. We could define arbritrary properties but this article mainly deals with properties derived from the so called \emph{type class laws}. In Haskell type class laws are exhibited by \glspl{typeclass}. For example, the property formed by equation \ref{eq:firstfunctorlaw} is type class law.

Type classes are a concept to describe the behavior of a type. 
They are very similar to interfaces. If we want to make a type an instance of a type class, we have to implement the functions given by the \glspl{typesignature} of the type class. Figure \ref{fig:typeclassrelation} illustrates the relation between values, types and type classes in Haskell.

An instance of a type class must obey the corresponding type class laws. Laws allow us to make assumptions about the behavior of the program and reason about the code. Properties of existing code allow us to rely on expected behavior and deduce further properties for new code. For example, if the property formed by equation \ref{eq:firstfunctorlaw} holds, we can rest assure that \verb|fmap| only applies the function to the items in the container and does nothing more, like changing the structure. 

Type class laws aren't enforced by Haskell. We need to verify them yourself when implementing a instance. We can verify the correctness of these equations using equational reasoning to prove that the instances we build are sound. 

Section \ref{sec:typeclasses} will give a short introduction to type classes. The example proof in section \ref{sec:example} will verify a type class law.

\tikzset{multi  attribute/.style={attribute ,double  distance=1.5pt}}
\tikzset{derived  attribute/.style={attribute ,dashed}}
\tikzset{total/.style={double  distance=1.5pt}}
\tikzset{every  entity/.style={draw=orange , fill=orange!20}}
\tikzset{every  attribute/.style={draw=MediumPurple1, fill=MediumPurple1!20}}
\tikzset{every  relationship/.style={draw=Chartreuse2, fill=Chartreuse2!20}}

\begin{figure}
\centering
\begin{tikzpicture}[node distance=10em]
  \node[entity](type){type};
  \node[entity](typeclass)[right of=type]{type class};
  \node[entity](value)[left of=type]{value};
  \draw (type) -- (typeclass) node [very near end, above=1pt] {*} node [very near start, above=1pt] {1};
\draw (value) -- (type) node [very near end, above=1pt] {1} node [very near start, above=1pt] {*};
  
\end{tikzpicture}
\caption{Relation between value, type and type class}
\label{fig:typeclassrelation}
\end{figure}
\subsection{Overview}

This article will illustrate in detail how equational reasoning works in practice using the type class laws of the type class \verb|Monoid| as the running example. Section \ref{sec:example} will walk through the process of proofing that the type class implementation obeys the first monoid law.  In order understand every step of the process the first three sections will explain the fundamental concepts used in section \ref{sec:example}. The article doesn't require knowledge of type classes and monoids. 
Section \ref{sec:typeclasses} will give a short introduction to type classes with a few examples and section \ref{sec:monoid} describes monoids. In section \ref{sec:equationalreasoning} we will give an introduction to equational reasoning in general before we proof the correctnes of the example in section \ref{sec:example}.

This article give answers to the following questions:
\begin{itemize}
\item What are desired properties of your program? Where do the come from? (type classes) Why are they useful? (covered in section \ref{sec:typeclasses})
\item What is equational reasoning? What is the difference between testing and proofing? How do we use equational reasoning to verify properties of a program? (covered in section \ref{sec:equationalreasoning}
\end{itemize}

