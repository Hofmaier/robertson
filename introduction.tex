\section{Introduction}
\label{sec:intro}

Type classes are a concept to describe behavior. In addition, they are excpectet to exhibit certain properties. This properties are called laws. 
Laws allow the programmer to make assumption about the behavior and reason about the code. We can rely on certain properties like associativity for example.

The compiler doesn't verifiy if a given property holds. We have to verify the laws with formal verfification. Formal verification is a tool use for software application with the aim of providing a high reliability. The first formally verified micro kernel, seL4 was verified using Haskell. Haskell was used because programmers can reason about the source code \cite{Klein09}. Correctness is specified in form of equations. The source code must satisfy this equations. Programmers can use equational reasoning to verify the correctness of these equations.

Purely functional programming languages are easier to reason about as imperative or object-oriented languages. All examples in this articel a written in Haskell.

In this article we first describe type classes with a few examples. Then we show that eqational reasoning can be used to verify properties of a haskell program. In a example we will prove that the monoid laws holds for new type. The next section shows that equational reasoning is used in a common library like pipes.


Some programmers care about the correctness of their software. It's hard to write a correct program at the first attempt. 
Hence, verification of a program improves the reliability of the software. 