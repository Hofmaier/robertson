\section{Example proof for Monoid Laws}
\label{sec:example}
In this example we use equational reasoning (see section \ref{sec:equationalreasoning}) to prove the monoid laws (see section \ref{sec:monoid}) for a new type.
The following example is taken from a blog post of Gabriel Gonzales \cite{gonzales14}. 

Suppose we want to build a plugin system. A plugin in our example is a \verb|IO| action, that takes a \verb|Char| value and does some work with it (e.g. log to a file, potentially with side effect). We demand the following requirements:
\begin{itemize}
\item We want to be able to add an arbitrary number of plugins.
\item The plugins should be composable.
\item The order we add the plugins must not matter.
\end{itemize}

A monoid instance satisfies all listed requirements. We could use the \verb|mappend| operator to compose several plugins. Hence, if we can prove that the plugins satisfy the monoid laws, we can combine them, using the monoid function \verb|mappend| and we are able to add plugins without concerning about the order of evaluation. In addition they are easier to use because, they will behave as expected.

The following listing shows the use of a plugin. The \verb|logto| function is a plugin. The program will read a character \verb|c| from the command line and apply the given plugin to \verb|c|.

\begin{verbatim}
main = do
    handleChar <- logto
    c <- getChar
    handleChar c
\end{verbatim}

To append additional plugins, e.g. \verb|print2stdout|, we compose a new monoid with the \verb|mappend| function.
\begin{verbatim}
handleChar <- mappend logto print2stdout
\end{verbatim}
The definition of \verb|logTo| is shown in the following listing: 

\begin{verbatim}
logTo :: IO ( Char -> IO ())
logTo = do
    handle <- openFile "log.txt" WriteMode
    return (hPutChar handle)
\end{verbatim}

The plugins are of type \verb|IO ( Char -> IO ())|. Hence we have to provide a monoid instance implementation for the type \verb|IO ( Char -> IO ())|. Instead of writing a specialized instance for \verb|IO ( Char -> IO ())|, we use the general implementation of section \ref{sec:monoid}.
The instance implementation is repeated for convenience:

\begin{lstlisting}[caption={Monoid instance},label={lst:monoidinstance}]
instance (Applicative f, Monoid b) => Monoid (f b) where
    mempty = pure mempty

    mappend = liftA2 mappend
\end{lstlisting}
The generalization has the advantage, that we have to prove the type class law only once for all types that match the general type declaration. The process of verifying a program is cumbersome and time consuming. Generalization of proofs is desirable.

We will first describe, why the instance implementation of listing \ref{lst:monoidinstance} makes the type \verb|IO ( Char -> IO ())| part of the \verb|Monoid| type class. Then we prove that the monoid laws from section \ref{sec:monoid} hold for the definition in listing \ref{lst:monoidinstance}. We will assume that the instance implementation of listing \ref{lst:monoidinstance} is a monoid in the first part.

\subsection{Generalization of the plugin type }
\label{sec:generalization}

In order to use the instance implementation of listing \ref{lst:monoidinstance} the type \verb|IO| has to be part of the type class \verb|Applicative| and the type \verb|Char -> IO ()| has to be a \verb|Monoid|. 
Hence we have to check if \verb|Char -> IO ()| is a monoid. We use the instance definition of listing \ref{lst:monoidinstance} again and say: \verb|Char -> IO ()| is a monoid if the type \verb|(->) Char| (that's the type of a haskell function) is part of the \verb|Applicative| type class and the type \verb|IO ()| is a monoid. We repeat the same argument for \verb|IO ()|.

Here is an overview of all the steps of the argumentation:
\begin{enumerate}
\item show that  \verb|IO ( Char -> IO ())| is of type 
\begin{verbatim}
(Applicative f, Monoid b) => Monoid (f b)
\end{verbatim}
\item show that \verb|Char -> IO ()| is a monoid
\item show that \verb|IO ()| is a monoid
\item show that \verb|()| is a monoid
\end{enumerate}

We show the required properties in reversed order.

\begin{etaremune}
\item The standard library provides a monoid instance for \verb|()| \cite{monoid}. In this article we will believe the fairy tale of abstraction and assume that the implementation of the standard library satisfies the monoid laws.
\item \verb|IO ()| is a monoid, because \verb|IO| is part of the \verb|Applicative| type class \cite{control.applicative} and \verb|()| is a monoid if the implementation from listing \ref{lst:monoidinstance} satisfies the monoid laws.
\item The type \verb|(->) r| (that's the type of haskell functions) is a part of the \verb|Applicative| type class \cite{control.applicative}.
 \verb|Char -> IO ()| is a monoid because \verb|(->) Char| is a applicative and \verb|IO ()| is a monoid if the implementation from listing \ref{lst:monoidinstance} satisfies the monoid laws.
\item The compiler will use the instance implementation for \verb|Monoid| type class from listing \ref{lst:monoidinstance} for the type \verb|IO (Char -> IO ())| because it matches the type
\begin{verbatim}
(Applicative f, Monoid b) => Monoid (f b)
\end{verbatim}

\end{etaremune} 

Notice that we rely heavily on the assumption that listing \ref{lst:monoidinstance} satisfies the monoid laws. The next section will prove that the implementation is correct.

\subsection{Proof}
\label{sec:exampleproof}

In this section we will show that the implementation in listing \ref{lst:monoidinstance} satisfies the left identity law of the \verb|Monoid| type class (see section \ref{sec:monoid}).
The left identity law demands that:
\begin{verbatim}
mappend mempty x = x
\end{verbatim}

We will use equational reasoning (see section \ref{sec:equationalreasoning}) to show that the left-hand side is equal to \verb|x|. First we will use the definitions of \verb|mappend| and \verb|mempty| of listing \ref{sec:equationalreasoning} to substitute the left-hand side. Furthermore we look-up the definition of \verb|liftA2| in the source code \cite{control.applicative} to evaluate the expression. 
\verb|liftA2| is defined as follows
\begin{verbatim}
liftA2 f x y = (pure f <*> x) <*> y
\end{verbatim}

\begin{verbatim}
mappend mempty x                           -- def. mappend
= liftA2 mappend mempty x                  -- def. mempty
= liftA2 mappend (pure mempty) x           -- def. liftA2
= (pure mappend <*> pure mempty) <*> x
\end{verbatim}

To resolve this expression further, we use the laws described in section \ref{sec:applicatives}.
One law of the \verb|Applicative| type class says:
\begin{verbatim}
pure f <*> pure x = pure (f x)
\end{verbatim}
We can use this property substitute the left-hand side. Next we write \verb|mappend mempty| as lambda function \verb|\a -> mappend mempty a| and use the monoid law
\begin{verbatim}
mappend mempty x = x
\end{verbatim}
to simplify the expression. In the last step we use the first applicative law
\begin{verbatim}
pure id <*> v = v
\end{verbatim}
 to rewrite the expression as \verb|x|.
\begin{verbatim}
(pure mappend <*> pure mempty) <*> x    -- 3. applicative law
= pure (mappend mempty) <*> x           -- transform to lambda
= pure (\a -> mappend mempty a) <*> x   -- 1. monoid law 
= pure (\a -> a) <*> x                  -- a -> a = id
= pure id <*> x                         -- 1. applicative law
x
\end{verbatim}

That completes the proof.

The example demonstrated several ideas:
\begin{itemize}
\item Type classes allow us to generalize definitions. A prove for the generalization is valid for all specializations.
\item To prove a type class law we can use equational reasoning.
\item Type class laws (or properties) allow to prove further properties.
\end{itemize}

