\section{Example proof for Monoid Laws}
\label{sec:example}
In this section we use equational reasoning (see section \ref{sec:equationalreasoning}) to prove the left identity law of the monoid type class (see section \ref{sec:monoid}) for a new type.
The following example is taken from a blog post of Gabriel Gonzales \cite{gonzales14}. We extended the example  with an addidional plugin implementation.

Suppose we want to build a plugin system. A plugin in our example is an \verb|IO| action that takes a \verb|Char| value and does some work with it (e.g. log to a file, potentially with side effects). Hence, a plugin is of type \verb|IO ( Char -> IO ())|. Listing \ref{lst:logtoplugin} shows the definition of the plugin \verb|logto|. It writes the first argument of the \verb|IO| action of type \verb|Char| to a file named \verb|log.txt|. It would be reasonable to parameterize the filename but in order to simplify the example the filename is hard-coded.

\begin{lstlisting}[caption={Definition of a plugin. It writes a character to a file.}, label={lst:logtoplugin}]
import System.IO
logto :: IO ( Char -> IO ())
logto = do
  handle <- openFile "log.txt" WriteMode
  return (hPutChar handle)
\end{lstlisting}

Listing \ref{lst:applyingaSinglePlugin} shows an application of the \verb|logto| plugin in the \verb|main| function. We bind the function of the plugin to the variable \verb|handleChar| and call it later with a character read from the standard input as argument (\verb|handleChar c|).

\begin{lstlisting}[caption={Calling the logto IO action in main}, label={lst:applyingaSinglePlugin}]
main = do
    handleChar <- logto
    c <- getChar
    handleChar c
\end{lstlisting}

Listing \ref{lst:print2stdout} shows an additional plugin that prints the character to the standard output. 

\begin{lstlisting}[caption={Definition of the plugin print2stdout}, label={lst:print2stdout}]
print2stdout :: IO ( Char -> IO ())
print2stdout = return (putChar)
\end{lstlisting}
In order to extend the program from listing \ref{lst:applyingaSinglePlugin}, we want to be able to compose several plugins.
Listing \ref{lst:composition} gives an example of the composition of the plugins \verb|logto| and \verb|printstdout|. We apply the \verb|mappend| function from the \verb|Monoid| type class to \verb|logto| and \verb|print2stdout|. Both arguments are of type \verb|IO ( Char -> IO ())|. The return value is of type \verb|IO ( Char -> IO ())| again. Of course listing \ref{lst:composition} only works if \verb|mappend| is implemented for the type \verb|IO ( Char -> IO ())|. We will give an implementation later in this section.

\begin{lstlisting}[caption={Composition of two plugins}, label={lst:composition}]
composedPlugin :: IO ( Char -> IO ())
composedPlugin = mappend logto print2stdout
\end{lstlisting}

In addition, the order in which we add the plugins must not matter. The behavior of the plugins \verb|composed1| and \verb|composed2| from listing \ref{lst:order} should be the same.

\begin{lstlisting}[caption={The order of the plugins must not matter.}, label={lst:order}]
composed1 = mappend logto print2stdout
composed2 = mappend print2stdout logto
\end{lstlisting}

The monoid laws state that \verb|mappend| must be associative. Hence, if we can prove that our implementation of \verb|mappend| satisfies the monoid laws, we can combine them, using the monoid function \verb|mappend| and we are able to add plugins without concerning about the order of evaluation. In addition, they are easier to use because they will behave as expected.

The plugins are of type \verb|IO ( Char -> IO ())|. Instead of writing a specialized instance for \verb|IO ( Char -> IO ())|, we use the general implementation of section \ref{sec:monoid}.
The instance implementation is repeated for convenience:

\begin{lstlisting}[caption={Monoid instance},label={lst:monoidinstance}]
instance (Applicative f, Monoid b) => Monoid (f b) where
    mempty = pure mempty

    mappend = liftA2 mappend
\end{lstlisting}
The generalization has the advantage, that we have to prove the type class law only once for all types that match the general type declaration. The process of verifying a program is cumbersome and time consuming. Hence, the generalization of proofs is desirable.

We will verify that the definition \ref{lst:monoidinstance} satisfies the left identity law of the monoid type class in two steps. First, we describe, why the instance implementation of listing makes the type \verb|IO ( Char -> IO ())| part of the \verb|Monoid| type class. Then we prove that the monoid laws from section \ref{sec:monoid} hold for the definition in listing \ref{lst:monoidinstance}. We will assume that the instance implementation of listing \ref{lst:monoidinstance} is a monoid in the first step.

\subsection{Generalization of the plugin type }
\label{sec:generalization}

In order to use the instance implementation of listing \ref{lst:monoidinstance}, the type \verb|IO| has to be part of the type class \verb|Applicative| and the type \verb|Char -> IO ()| has to be a \verb|Monoid|. 
Hence, we have to check if \verb|Char -> IO ()| is a monoid. We use the instance definition of listing \ref{lst:monoidinstance} again and say: \verb|Char -> IO ()| is a monoid if the type \verb|(->) Char| (that's the type of a Haskell function) is part of the \verb|Applicative| type class and the type \verb|IO ()| is a monoid. We repeat the same argument for \verb|IO ()|.

Here is an overview of all the steps of the argumentation:
\begin{enumerate}
\item Show that  \verb|IO ( Char -> IO ())| is of type 
\begin{verbatim}
(Applicative f, Monoid b) => Monoid (f b)
\end{verbatim}
\item Show that \verb|Char -> IO ()| is a \verb|Monoid|.
\item Show that \verb|IO ()| is a \verb|Monoid|.
\item Show that \verb|()| is a \verb|Monoid|.
\end{enumerate}

We show the required properties in reversed order.

\begin{etaremune}
\item The standard library provides a \verb|Monoid| instance for \verb|()| \cite{monoid}. All \verb|Monoid| instances in the standard library obey the monoid laws \cite{Lipovaca}. Therefore, \verb|()| must form a \verb|Monoid|.

\item If the implementation from listing \ref{lst:monoidinstance} satisfies the monoid laws then \verb|IO ()| is a monoid because \verb|IO| is part of the \verb|Applicative| type class \cite{control.applicative} and \verb|()| is a \verb|Monoid|.

\item The type \verb|(->) r| (that's the type of Haskell functions) is a part of the \verb|Applicative| type class \cite{control.applicative}.
If the implementation from listing \ref{lst:monoidinstance} satisfies the monoid laws then \verb|Char -> IO ()| is a \verb|Monoid| because \verb|(->) Char| is an \verb|Applicative| and \verb|IO ()| is a \verb|Monoid|.

\item The compiler will use the instance implementation for \verb|Monoid| type class from listing \ref{lst:monoidinstance} for the type \verb|IO (Char -> IO ())| because it matches the type
\begin{verbatim}
(Applicative f, Monoid b) => Monoid (f b)
\end{verbatim}

\end{etaremune} 

Notice that we rely heavily on the assumption that listing \ref{lst:monoidinstance} satisfies the monoid laws. The next section will prove that the implementation is correct.

\subsection{Proof}
\label{sec:exampleproof}

In this section we will show that the implementation in listing \ref{lst:monoidinstance} satisfies the left identity law of the \verb|Monoid| type class (see section \ref{sec:monoid}).
The left identity law demands that:
\begin{verbatim}
mappend mempty x = x
\end{verbatim}

We use equational reasoning (see section \ref{sec:equationalreasoning}) to show that the left-hand side is equal to \verb|x|. First we use the definitions of \verb|mappend| and \verb|mempty| of listing \ref{sec:equationalreasoning} to substitute the left-hand side. Furthermore we lookup the definition of \verb|liftA2| in the source code \cite{control.applicative} to evaluate the expression. 
\verb|liftA2| is defined as follows
\begin{verbatim}
liftA2 f x y = (pure f <*> x) <*> y
\end{verbatim}

\begin{verbatim}
mappend mempty x                           -- def. mappend
= liftA2 mappend mempty x                  -- def. mempty
= liftA2 mappend (pure mempty) x           -- def. liftA2
= (pure mappend <*> pure mempty) <*> x
\end{verbatim}

To resolve this expression further, we use the laws described in section \ref{sec:applicatives}.
One law of the \verb|Applicative| type class says:
\begin{verbatim}
pure f <*> pure x = pure (f x)
\end{verbatim}
We can use this property to substitute the left-hand side. Next we write \verb|mappend mempty| as lambda function \verb|\a -> mappend mempty a| and use the monoid law
\begin{verbatim}
mappend mempty x = x
\end{verbatim}
to simplify the expression. In the last step we use the first applicative law
\begin{verbatim}
pure id <*> v = v
\end{verbatim}
 to rewrite the expression as \verb|x|.
\begin{verbatim}
(pure mappend <*> pure mempty) <*> x    -- 3. applicative law
= pure (mappend mempty) <*> x           -- transform to lambda
= pure (\a -> mappend mempty a) <*> x   -- 1. monoid law 
= pure (\a -> a) <*> x                  -- a -> a = id
= pure id <*> x                         -- 1. applicative law
x
\end{verbatim}

That completes the proof.

The example demonstrated several ideas:
\begin{itemize}
\item Type classes allow us to generalize definitions. A proof for the generalization is valid for all specializations.
\item To prove a type class law we can use equational reasoning.
\item Type class laws (or properties) allow to prove further properties.
\end{itemize}

