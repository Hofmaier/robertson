\begin{abstract}

One of the appeals of a purely functional programming language, like Haskell, is that it is substantially easier to verify the correctness of a program. The reason is referential transparency. The fact that function definitions in Haskell are equalities, allows us to reason about properties of a program by substituting expressions with equal expressions. This process is called equational reasoning.

Some type classes in Haskell exhibit properties, the so called “type class laws”, that every corresponding type class instance is expected to obey. These properties ensure certain desirable behavior (e.g. associativity), or they can be used to conclude further properties. Type class laws aren't enforced by the Haskell compiler, so we need to check them ourselfs.
 
This report describes the technique equational reasoning and how it can be used to prove that a type class implementation obeys the corresponding type class laws. Further, it gives a brief description of the concept of a type class and describes the type class \verb|Monoid| and its laws in more detail.
Finally, we walk through an application example. We use equational reasoning to proof a type class law for a simple plugin system written in Haskell. 
\end{abstract}
