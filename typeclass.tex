\section{Type classes}
\label{sec:typeclasses}

The first part of this section will give a short description of the concept of a type class. The second part will describe monoids in general and the type class \verb|Monoid|. The example in section \ref{sec:example} will prove a monoid type class law. 

The concept of a type class was introduced as a construct that supports overloaded functions and \emph{\gls{adhoc_polymorphism}} \cite{Wadler}. Overloaded functions can be used with a variety of types, but with different definitions for the different types. For example, the function calls \verb|show 1| and \verb|show "hello"| use different \glspl{function-definition}. The function \verb|show 1| is of type \verb|Int -> String| and \verb|show "hello"| is of type \verb|String -> String|. The definition used depends on the type of the argument.  The next section will describe \gls{adhoc_polymorphism} and the relation to type classes in more detail.

\subsection{Polymorphism and type classes}
\label{sec:polymorphism}
%%In this section we will describe relation between type classes and polymorphism.
There are two types of polymorphism in Haskell \cite{Cardelli}: Parametric polymorphism and \gls{adhoc_polymorphism}. Type classes are used for \gls{adhoc_polymorphism}.
\begin{description}
\item[Parametric polymorphism] refers to functions which work over more than one type. For example, the library function \verb|length| returns the length of a list \verb|[]| that contains items of arbitrary type. It can be used to calculate the length of a list of integers, a list of strings, a list of booleans, etc. There are no constraints. The type of \verb|length| is
\begin{verbatim}
length :: [t] -> Int
\end{verbatim}
\verb|t| is a \emph{type variable}. That is, for any type \verb|t| the function \verb|length| has type \verb|[t] -> Int|. A type that contains a type variable is called \emph{polymorphic}. Hence, \verb|length| is a polymorphic type.
The \verb|length| function can be used with any type but it has a \emph{single} definition.  At compile time, the type variables are substituted with a concrete type. For example
\verb|[Int] -> Int|. 
\item[Ad hoc polymorphism] is a synonym for function overloading or operator overloading. An overloaded function uses different function definitions depending on the types of the arguments. Suppose we want to define a function that converts a list containing items of arbitrary type (\verb|[t]|) to a string. We would write a function with the following \gls{typesignature}:
\begin{verbatim}
showlist :: [t] -> String
\end{verbatim}
It takes a list of arbitrary type \verb|t| and returns a string. The definition could look like this:
\begin{verbatim}
showlist [] = ""
showlist (x:xs) = show x ++ showlist xs
\end{verbatim}
We need a way to make sure that the function \verb|show| is defined for the type of the value \verb|x|. \verb|show| can't be a polymorphic type because the conversion depends on the type. There's no single definition that can convert an arbitrary type to a string. 

There's a set of types. \verb|show| is defined over all members of this set. This set is called a type class. The type class \verb|Show| for example contains all types that can be converted to a string with the function \verb|show|. In order to prevent the application of the function \verb|showlist| with an argument that isn't a member of the type class \verb|Show|, we must constraint the type variable in the type signature declaration \verb|t|:
\begin{verbatim}
showlist :: Show t => [t] -> String
\end{verbatim}
\end{description}

Functions declared by the type class are defined over all members of the type class. And certain type classes exhibit properties that every definition of the corresponding functions must obey.

\subsection{Monoid}
\label{sec:monoid}

In mathematics, a \gls{monoid} is an algebraic structure with single associative binary operation and an identity element. Monoids are semigroups with identity \cite{wiki:monoid} \cite{renshaw}. Several elements of a monoid can always be reduced to a single element by applying the corresponding binary operator. It doesn't matter in which order we apply the operator, the result is always the same. This is called \emph{\gls{associativity}}. The set of elements has an identity element. For example, the set of natural numbers $\mathbb{N}$ form a monoid under multiplication.  The number 1 is the identity element. Multiplication of the identity and any other number $x$ results always in $x$.

In Haskell there is a type class for monoids. Types that form a monoid can become part of the \verb|Monoid| type class. For example, the type list \verb|[]| forms a monoid. Two values of type list can always concatenated to another list with the \verb|++| operator. The empty list \verb|[]| is the identity element. 

The example in section \ref{sec:example} shows a plugin system, that contains a monoid. Plugins can be composed with a binary operator. An arbitrary number of plugins can be composed to a single plugin. The order the plugins are added doesn't matter (\gls{associativity}). 

\subsubsection{Functions of the type class monoid}

Members of the type class \verb|Monoid| have to implement the functions \verb|mempty| and \verb|mappend| amongst others (see appendix \ref{sec:monoiddefinition} for a complete declaration).
These functions have the following type signature declaration.
\begin{verbatim}
    mempty :: m
    mappend :: m -> m -> m
\end{verbatim}
The type variable \verb|m| is the type of the corresponding monoid.
\verb|mempty| returns the identity value. \verb|mappend| is the binary function that takes two values of the same type and returns another value of that type. 

The type class \verb|Monoid| exhibits several laws. We will only describe the one that we will prove in the example of section \ref{sec:example}, the \emph{left identity law}.
When making monoid instances, we need to make sure that \verb|mempty| acts like the identity with respect to the \verb|mappend| function. This property can be expressed with the following equation:
\begin{equation}
  \label{eq:firstmonoidlaw}
  \text{mappend}(\text{mempty}, x) = x
\end{equation}
Equation \ref{eq:firstmonoidlaw} states that \verb|mempty| has to behave like the identity with respect to \verb|mappend|. When \verb|mappend| is applied with the identity and an other element \verb|x| of the monoid, we get back \verb|x|.

\subsubsection{Example monoid implementation}

There is a useful property of the \verb|Applicative| type class with respect to the \verb|Monoid| type class (the \verb|Applicative| type class is described in more detail in the appendix, section \ref{sec:applicatives}). The example in section \ref{sec:example} will use this property. If \verb|f| is an \verb|Applicative| and \verb|b| is a \verb|Monoid| then \verb|f b| is also a \verb|Monoid|. If a type is part of the \verb|Applicative| type class and the type contains a \verb|Monoid| we can create a \verb|Monoid| instance with the implementation in listing \ref{lst:monoidinstance1}. Figure \ref{fig:applicative_monoid} illustrates this property.

\begin{figure}
  \centering
     \includegraphics[width=0.7\textwidth]{monoid}
  \caption{An Applicative that encapsulates a monoid is a monoid}
  \label{fig:applicative_monoid}
\end{figure}

\lstset{
basicstyle=\ttfamily,
columns=fullflexible,
keepspaces=true,
captionpos=b
}
\begin{lstlisting}[caption={{\ttfamily Monoid} instance implementation of {\ttfamily IO}},label={lst:monoidinstance1}]


{-# LANGUAGE FlexibleInstances #-} 
import Data.Monoid
import Control.Applicative 

instance (Applicative f, Monoid a) => Monoid (f a) where
    mempty = pure mempty
    mappend = liftA2 mappend
\end{lstlisting}

\verb|mempty| is of type \verb|f a| . Hence \verb|pure mempty| has to be of type \verb|f a|.
As \verb|f| is an \verb|Applicative|, it implements \verb|pure|. The type of \verb|pure| is \verb|a -> f a| (see appendix \ref{sec:applicatives}). We call \verb|pure| with \verb|mempty| of type \verb|a|. We know that \verb|a| is part of \verb|Monoid| because of the type constraints. The compiler will use \verb|mempty| of \verb|a|. \verb|liftA2| is an utility function of \verb|Applicative|. It encapsulates the \verb|mappend| function in an applicative functor. 

In section \ref{sec:example} we prove that the \gls{function-definition} of listing \ref{lst:monoidinstance1} obeys the first monoid law formed by equation \ref{eq:firstmonoidlaw} with  the verification technique equational reasoning.


