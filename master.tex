\documentclass[twoside, a4paper]{article}
\usepackage{amsmath}
\usepackage{float}
\usepackage{graphicx}
\usepackage{url}
\usepackage{etaremune}
\newfloat{program}{thp}{lop}
\floatname{program}{Listing}
\author{Lukas Hofmaier}
\title{Type class laws}

\begin{document}
\maketitle
\tableofcontents

\section{Introduction}
\label{sec:This}

When I started learning functional programming I often read about the advandages of pure functional programming languages. One advandage is that they are great for \gls{verification} and equational reasoning \cite{Wadler87}.
I didn't know precisely what equational reasoning is and how it can be used to verify certain properties of a program. I also didn't know what kind of properties of a program can be verified with equational reasoning.
This article describes the verification method equational reasoning by example. 

\subsection{Why use equational reasoning?}

Equational reasoning can be used as a formal method for software verification. It possible to proof the correctness of a program. Correctness means that specified properties of the program hold for an arbitrary input. 

A property of a program can be represended as an equation. For example

\begin{equation}
  \label{eq:firstfunctorlaw}
\text{fmap } \text{id}  =  \text{id}  
\end{equation}

is a property of the function \verb|fmap|. The function \verb|fmap| applies it's first parameter to every element in a container (mapping). The function \verb|id| is the identity function. This property says that appliing the function \verb|fmap| with the argument \verb|id| is the same as appling the function \verb|id|. Put it another way, mapping the identiy function over every item in a container has no effect. Given a function definition of \verb|fmap| we can proof that this property holds using equational reasoning.

The goal of equational reasoning is to provide high reliability. 
For example, the first formally verified micro kernel, seL4 was verified using formal verification to ensure proper security functionality and security assurance \cite{Klein09}.
Another example is the streaming library pipes. The author, Gabriel Gonzales, used equational reasoning to prove the correctness of the library.

\subsection{Why are functional programming languages great for equational reasoning?}

The reason why we can use equational reasoning in functional programming is \gls{referential_transparency}. A piece of code can be replaced by its value; this is true in purely functional languages. The absence of side effects allow us the replace the left-hand side of a definition with the right-hand side and vice verca. For example if you have the following Haskell definition:
\begin{verbatim}
x = 23
\end{verbatim}
That means wherever you see an \verb|x| in your program you can substitute it with \verb|23|.
This isn't possible in programs written in an imperative language because they are dependent on their context. For example if we have
\begin{verbatim}
x = 23
\end{verbatim}
we must not substitute \verb|x| with \verb|23| in your program. Arbritrary integers could be assigned to the variable \verb|x| in later statements of the program.

The programming language Haskell is a purely functional language, which means that in general, functions in Haskell do not have side effects. Hence, we can verify Haskell program using equational reasoning. For this reason all examples in this article a written in Haskell.

\subsection{What properties will we discuss?}

Equational reasoning is used to proof if certain properties hold. In Haskell so called type classes dictate properties. Type classes are a concept to describe behavior of a type. They are very similar to interfaces but some type classes exihibit certain properties that every implementation must obey. Properties of type classes are called type class laws. Laws allow us to make assumption about the behavior of the program and reason about the code. Properties of existing code allow us to rely on expected behavior and deduce further properties for new code. Section \ref{sec:typeclasses} will give a short introduction to type classes. The example proof in section \ref{sec:example} will verify a type class law.

\subsection{Overview}

This article will illustrate in detail how equational reasoning works in practice using the type class laws of monoid as the running example. Section \ref{sec:example} will walk through the process of proofing a certain property for a plugin system. We will proof that the type of the plugins obej the so called monoid laws with the methode equational reasoning.  The monoid laws are specified by the type class monoid. In order understand  example in \ref{sec:example} the first three sections will explain the fundamental concepts used in section \ref{sec:example}. The article doesn't require knowledge of type classes ands monoids. 
Section \ref{sec:typeclasses} will give a short introduction to type classes with a few examples and section \ref{sec:monoid} describes monoids. In section \ref{sec:equationalreasoning} we will give an introduction to equational reasoning.

This article give answers to the following questions:
\begin{itemize}
\item What are desired properties of your program? Where do the come from (type classes)? Why are they useful? (section \ref{sec:typeclasses}
\item What is equational reasoning? What is the difference between testing and proofing? How do we use equational reasoning to verify properties of a program?
\end{itemize}


 In addition, they exhibit certain properties. These properties are called laws. 


For example the typeclass \verb|Functor| makes sure that we can apply the function \verb|fmap| with a type that is part of the \verb|Functor| type class. The type class dictates the behavior. All functors are expected to exhibit certain kinds of properties. These properties guarante that the function \verb|fmap| only applies a function to all values inside the functor and doesn't change the structure or the context. The expression
\begin{verbatim}
dontchangecontext = do line <- fmap reverse getLine
                       putStrLn $ " "
\end{verbatim}
applies \verb|reverse| to the result of the IO function \verb|getLine| without changing the the inputstream. 
If we can rely on known properties of your program, we minimise unexpected behavior.

Type class instances from the standard library obey the type class laws. If we write new instances, it's our responsibility to check if the type class laws hold.
It's in the responsibility of the programmer to check if the program behaves correctly. For most programmers it's hard to write a correct program at the first attempt. This leaves us with the question of how to verify the type class laws. There are several methods to verify software. In this article we will look at technique of equational reasoning to prove type class laws. 

Properties are specified in form of equations. The source code must satisfy this equations. 



An equation is a statement that two expressions are equal (such as $1 + 3 = 4$). And reasoning is the process of thinking about something in a logical way in order to form a conclusion. The methods implies that we show correctness by showing that two expression are equal. 


\section{Type classes}
\label{sec:typeclasses}

This section will give a short description of the concept type class. It will explain the relation to polymorphism and describe type classes \verb|Functor|, \verb|Applicative| and \verb|Monoid| in more detail as these type classes are important for the example in section \ref{sec:example}.

The concept of a type class was introduced as a construct that supports operator overloading and ad-hoc polymorphism \cite{Wadler}.

A type class is like an interface. It contains the function declarations. A type becomes an instance of a type class when it defines all required functions of the type class. When a type is an instance of a type class, we can make certain assumptions about behavior of the type. In contrary to interfaces, type classes aren't types. A value can have the type of an interface but not of a type class.

The concept of a type class is explained by example with the function \verb|show| from the \verb|Prelude|-Library (\verb|Prelude| is a module and part of the standard). \verb|show| converts a given value of a type \verb|a| into a character string. The type of \verb|show| is

\begin{verbatim}
show :: Show a => a -> String
\end{verbatim}

The \verb|Show a| before the \verb|=>| is a type class constraint. \verb|a| is a type variable. A type that has one more type variables is called polymorphic \cite{hutton}. The signature means that \verb|show| takes something that implements the type class \verb|Show| and returns a string. \verb|Show| is a type class. It's possible to call \verb|show| with different types (e.g \verb|show 1|, \verb|show "hello"|). The compiler will lookup the correct definition for us as long the type of the first parameter is an instance of the type class \verb|Show|.  
Any type that implements \verb|Show| can be converted to a character string. Types in this class are \verb|Bool|, \verb|Char|, \verb|Int|, \verb|Float|, \verb|Double| etc.

The type class \verb|Show| is defined as follows:
\begin{verbatim}
class Show a where
    show :: a -> String
\end{verbatim}
The keyword \verb|class| defines a new type class. \verb|a| is the type variable. It represents the type that implements the type class (e.g. \verb|Int| or \verb|Bool|).

Once we have a type class we are able to create instances of that class. The following listing defines a type \verb|Person|. It has fields for name and email address. 
\begin{verbatim}
data Person = Person { name :: String
                       email :: String
                     }
\end{verbatim}

To make \verb|Person| an instance of \verb|Show| we provide a definition for \verb|show :: Person -> String|

\begin{verbatim}
instance Show Person where
    show (Person name _) = name
\end{verbatim}

There many other useful type classes in the standard library.

\begin{description}
\item[Ord] Types with an order relation implement \verb|Ord|.
\item[Eq] For types that can be equated.
\item[Read] Types that can be convertet from a string.
\end{description}

\subsection{Polymorphism and type classes}
\label{sec:polymorphism}
In this section we will describe relation between type classes and polymorhism.
There are two types of polymorphism in Haskell \cite{Cardelli}. Type classes are used for ad hoc polymorhism.
\begin{description}
\item[Parametric polymorphism] Refers to a type that contains type variables. For example the type of the function \verb|id| is 
\begin{verbatim}
id:: a -> a
\end{verbatim}
The \verb|id| function can be used with any type. There are no constraints. At compile time, the type variables are substituted with a concrete type. For example
\verb|Char -> Char|.
\item[Ad-hoc polymorphism] It's a synonym for function overloading or operator overloading. Polymorphic functions can be applied to values with different types. A polymorphic function uses different definitions (implementations) depending on the types of the arguments. If a type can be converted to a string, it can be given the type class \verb|Show|. The type \verb|Person| has to provide an implementation for the function \verb|show|. Applying \verb|show| to \verb|Person| results in a different behavior then applying \verb|show| to an \verb|Int|.
\end{description}

\subsection{Functor}
\label{sec:functor}

Functor is a type class for types, which can be mapped over. 

The type class declaration is shown in Listing \ref{fig:functordeclaration}.
The \verb|f| in the declaration is a type class constructor. Only type constructor can implement \verb|Functor| (\verb|Maybe|, \verb|[]|).

Functor defines all instances must implement the function \verb|fmap|.
\begin{figure}
  \centering
\begin{verbatim}
class Functor f where
    fmap :: (a -> b) -> f a -> f b
\end{verbatim}
  \caption{Functor type class declaration}
  \label{fig:functordeclaration}
\end{figure}

\verb|fmap| takes any function \verb|a -> b| and a value of type \verb|f a| and returns a value of type \verb|f b|. 
If \verb|f| is of type \verb|Maybe| \verb|Int| and the function of type \verb|Int -> String|, \verb|fmap| returns \verb|Maybe String|. 

\verb|fmap| applies a function to a value without altering its structure or context.

Instances of \verb|Functor| are:

\begin{description}
\item[List] \verb|map| for lists for is the same as \verb|fmap|.
\item[Either] \verb|Either e a| is a container. \verb|fmap| applies a function to \verb|a|.
\end{description}

Instances of the \verb|Functor| type class are expected to exhibit certain kinds of properties. This properties are called the functor laws.
The Haskell Compiler doesn't detect violations of the expected laws. All Functor instances in the standard library obey these laws.

A Functor instance has to satisfy the following laws \cite{Marlow_2010}.

\begin{description}
\item[Law 1] Mapping the identity function over a functor value, will not change the functor value. Formally
\begin{verbatim}
fmap id  ==  id
\end{verbatim}
\item[Law 2] The second law states that it doesn't matter if we compose two functions and them map them over a functor or if we first map one function over the functor and then map the other function. Formally
\begin{verbatim}
fmap (g . h) = fmap g . fmap h
\end{verbatim}
This is the same as \verb|fmap (g . h) = fmap g (fmap h)|
\end{description}

If we can prove that a type satisfies these laws, we can make assumptions about how the the type will act. And we know that \verb|fmap| only maps the function over the functor. It will not change the structure or the context of the functor.

If we know, that a type satisfy the laws, we are able to deduce further properties for our own types. All type class instances in the standard library satisfy the laws \cite{yorgey}.

\begin{figure}
  \centering
\begin{verbatim}
fmap id = id
fmap (g . h) = fmap g . fmap h
\end{verbatim}
  \caption{The Functor laws}
  \label{fig:functorlaws}
\end{figure}

\subsection{Applicative Functor}
\label{sec:applicatives}

Applicative functors are an abstract characterisation of an applicative style of effectful programming \cite{mcbride} \cite{control.applicative}

The \verb|Applicative| type class encapsulates the following idea. What if you have a function wrapped in a \verb|Functor| (e.g \verb|Maybe (Int -> Int -> Int)|) and you want to apply the function to another functor. 

The \verb|Applicative| type class is defined as follows.
\begin{verbatim}
class (Functor f) => Applicative f where
    pure :: a -> f a
    (<*>) :: f (a -> b) -> f a -> f b
\end{verbatim}

Applicatives are enhanced functors. They are able to map a function encapsulatet in a functor over another functor.

\begin{figure}
  \centering
     \includegraphics[width=0.9\textwidth]{functor_applicative}
  \caption{Comparison of test, property-based-testing and proof}
  \label{fig:property_validation}
\end{figure}


The following type are instances of the type class \verb|Applicative|

\begin{itemize}
\item Maybe
\item IO
\end{itemize}

\subsection{Monoid}
\label{sec:monoid}

Some types, let's say \verb|a|, have a binary function with the type declaration. 
\begin{verbatim}
f :: a -> a -> a
\end{verbatim}

The type \verb|a| has a value that serves as identity for the given function. For example 1 is the identity for the multiplication. Multiplicaiton with an other numer $x$ results always in $x$.

Several value of type \verb|a| can always be reduces to a single value. It doesn't matter in which order we apply the function, the result is always the same. This is called associativity.

If type \verb|a| has this behavior, it's a monoid and it can be an instance of the \verb|Monoid| type class.

The \verb|Monoid| type class is define as follows:
\begin{verbatim}
class Monoid m where
    mempty :: m
    mappend :: m -> m -> m
    mconcat :: [m] -> m
    mconcat = foldr mappend mempty
\end{verbatim}

The \verb|Monoid| type class is define in \verb|Data.Monoid| \cite{monoid}. 

From the type class definition we see that concrete type can be made instances of \verb|Monoid|. 

\verb|mempty| return the identity value. The binary function is \verb|mappend|. It takes to values of the same type and returns another value of that type. \verb|mconcat| takes a list of monoids and reduces them \verb|mappend| to a single value, applying mappend. It has a default implementation.

When making monoid instances, we need to make sure that \verb|mempty| acts like the identity with respect to the \verb|mappend| function and \verb|mappend| must be assocative. There are three monoid laws

\begin{enumerate}
\item \verb|mappend mempty x = x|
\item \verb|mappend x mempty = x|
\item \verb|(x `mappend y) `mappend z = x `mappend (y `mappend z)|
\end{enumerate}

The following Haskell types are \verb|Monoid| instances
\begin{description}
\item[List] The empty list \verb|[]| and \verb|++| (concatenation) form a monoid.
\item[Product and Sum] Numbers can be monoid with respect to multiplication or addidion. There are two monoid instances for \verb|Num|.
\item[Maybe] Can also be an instance of \verb|Monoid|.
\end{description}

An interesting property of the \verb|Applicative| type class with respect to the monoid type class is, if \verb|f| is an \verb|Applicative| and \verb|m| is a \verb|Monoid|, \verb|f b| is also a \verb|Monoid|. This is because we can make an \verb|Applicative| with the given property an instance of \verb|Monoid|.
\begin{program}
\begin{verbatim}
instance (Applicative f, Monoid b) => Monoid (f b) where
    mempty = pure mempty

    mappend = liftA2 mappend
\end{verbatim}
\label{lst:applicative2monoid}
\caption{Applicative property}
\end{program}

As \verb|f| is an applicative, it implements \verb|pure|. The \verb|mempty| function returns the identity value of \verb|b| in a default context. \verb|liftA2| encapsulates the \verb|mappend| function in a applicative functor. It's possible to prove this property with equational reasoning.

\section{Equational Reasoning}
\label{sec:equationalreasoning}

Verification is the process of checking if software does what it's specification demands. To verify a program, a specification is required. In the case of functions that implement an instance of the type classes of section \ref{sec:typeclasses}, the specification is defined in form of the type class laws.
This section will compare different verification techniques and describe the method equational reasoning by example using type class laws as specification. In section \ref{sec:example} equational reasoning will be applied to proof the property of listing \ref{lst:monoidinstance1} in section \ref{sec:typeclasses}.

There are several ways to check the behavior of a program. 
We will describe the difference with a simple example. Given the following property:

\begin{equation}
  \label{eq:reverse_prop}
\text{reverse} (\text{reverse } (xs)) = xs  
\end{equation}
Equation \ref{eq:reverse_prop} expresses, if we apply \verb|reverse| twice on the same list \verb|xs| we get back the original list \verb|xs|. \verb|reverse| is the inverse of \verb|reverse| (other functions, like \verb|id|, have this property too). Verification techniques allow us to check if equation \ref{eq:reverse_prop} holds. We describe three techniques. The first two, testing and property-based testing, are very common and the third is the topic of this article.

\begin{description}
\item[Testing] Run the program with a selected input and check if it behaves as expected. In order to check the behavior, a function evaluates both sides of equation \ref{eq:reverse_prop} and compares the values. The following listing shows an example test.

\begin{lstlisting}[caption={Function definition for testing},label={lst:testing}]
input = [1,2,3]
test_reverse :: Bool
test_reverse = reverse (reverse input) == input
\end{lstlisting}

The selected input is \verb|[1,2,3]|. \verb|test_reverse| evaluates to a Boolean expression to indicate if the property holds for the given input. It's necessary to run the program to evaluate \verb|test_reverse|.
An advantage of this method is, that the programmer doesn't have to define general properties. The specification is expressed with a concrete input value and a concrete output value. It's easier to think about a concrete input and the corresponding output value than to find general properties that a function must obey.
\item[Property-based testing] The input for the test program is generated randomly. The tests are executed by a tool (e.g. QuickCheck).
\item[Proof] A Proof can show that a property holds in all circumstances. To prove a property we use the technique equational reasoning. This technique requires knowledge of the function definition.
\end{description}

Figure \ref{fig:property_validation} compares the input coverage of the described methods. Testing checks if the program behaves correctly with one chosen point of the input space. Property-based testing checks the behavior at hundreds of randomly-generated points. Proof covers all possible cases of the possible input. It is the most reliable verification.

\begin{figure}
  \centering
     \includegraphics[width=0.9\textwidth]{testing}
  \caption{Comparison of testing, property-based-testing and proof}
  \label{fig:property_validation}
\end{figure}

\subsection{Reasoning about algebraic properties}

Equational reasoning is a method originally used in algebra. It's the process of proving a given property by substituting equal expressions.
For example, it's possible to show that the following property holds:
\begin{equation}
  \label{eq:sum}
  (x+a)(x+b) = x^2 + (a+b)x+ab
\end{equation}
To show that the equality holds, we have to transform the expression on the left-hand side  $(x+a)(x+b)$  to an equal expression on the right-hand side with the help of the basic algebraic properties of numbers (\gls{distributive_law}, \gls{commutative_law}) until we get $x^2 + (a+b)x+ab$  \cite{hutton}. 

In the first step (equation \ref{eq:firstalgebra} we use distributivity to expand the term on the left-hand side.
\begin{equation}
  \label{eq:firstalgebra}
  (x+a)(x+b) = x^2 + ax + xb + ab \text{     (use distributivity)}
\end{equation}
In the second step we use commutativity to substitute $xb$ with $bx$.
\begin{equation}
x^2 + ax + xb + ab = x^2 + ax + bx + ab \text{     (use commutativity)}
\end{equation}
In the last step we use distributivity to factorize $x$ and we get $x^2 + (a+b)x+ab$.
\begin{equation}
x^2 + ax + bx + ab = x^2 + (a + b)x + ab \text{     (use distributivity)}
\end{equation}
All we did, was substituting expression according the algebraic properties 

\subsection{Reasoning about Haskell programs}

A function definition in Haskell means that we can substitute the left-hand side with the right-hand side and vice versa. This is possible because Haskell is a purely functional language. Hence, we can use the same approach to prove that a property of a program written in Haskell holds, as we used to reason about mathematical expressions. 
For example, it's possible to show that the length of a list with one element is actually 1. This general property can be formed as boolean expression in Haskell.
\begin{verbatim}
length [x] == 1
\end{verbatim}
The property holds no matter what \verb|x| is. To show that, we use the \gls{function-definition} of \verb|length| as a general description of the behavior. Listing \ref{lst:lengthdefinition} shows the definition of \verb|length| \cite{hutton}.
\begin{lstlisting}[caption={Function definition of {\ttfamily length}},label={lst:lengthdefinition}]
length [] = 0
length (x:xs) = 1 + length xs  
\end{lstlisting}

In order to conclude that \verb|length [x] == 1| is always true, we substitute \verb|length [x]| until we get \verb|1|. Listing \ref{lst:lengthproof} shows the step by step substitution. It's a stepwise transformation of the expression \verb|length [x]| to \verb|1|.
\begin{lstlisting}[caption={step by step substitution of {\ttfamily length}},label={lst:lengthproof}]
length [x]         -- [x] is the same as x:[]
= length (x:[])    -- apply definition
= 1 + length []    -- apply definition
= 1 + 0            --  1 + 0 = 0
1
\end{lstlisting}
Function definitions are general description and we can use them to deduce other general properties by substituting equal expressions.

\subsection{Proof by structural induction}
\label{sec:induction}

If we apply simple substitution to a recursive function, we run into problems.
Consider the \gls{function-definition} of \verb|length| in listing \ref{lst:lengthdefinition}.
If we substitute \verb|length x| with the definition \verb|1 + length xs|, we end up substituting \verb|length x| forever. A way to verify recursive programs is to use proof by structural induction.  
Structural induction for proofing a property can be used for list or algebraic data types with a recursive constructor (e.g. Tree) \cite{Thompson}.

 The principle of induction states, that it is sufficient to prove a property $p$ for the base case and that $p$ is preserved by the inductive case \cite{Doets}. In order to prove $p$, two steps are required:
 \begin{description}
 \item[Base case] Prove $p(0)$ is true.
 \item[Induction step] Prove $p(n+1)$ if $p(n)$ (induction hypothesis) is true.
 \end{description}

Proof by induction is similar to writing a recursive function. Recursive functions use a base case (e.g. \verb|[]|, 0). 
If we use structural induction we proof the base case. We show that the property holds for a concrete input value (e.g. \verb|[]|, 0). 

In a recursive function definition we define \verb|f (x:xs)| and use \verb|f x| in the right-hand side. In the proof we show that $p(n+1)$ with the assumption that $p(n)$ holds.

We explain proof by structural induction with another example from \cite{Thompson}. We verify that the overall length of two concatenated lists $xs$ and $ys$, is the same as the sum of the length of $xs$ and the length of $ys$.  The ++-operator concatenates two lists.
\begin{equation}
  \label{eq:lengthprop}
  \text{length (xs ++ ys)} = \text{length(xs) + length(ys)}
\end{equation}
In order to verify property \ref{eq:lengthprop}, we need the function definitions for \verb|length| and \verb|(++)|.
Listings \ref{lst:lengthdefinition} and \ref{lst:concatdefintion} show he function definitions of \verb|length| and \verb|(++)|.

\begin{lstlisting}[caption={Haskell function definition of the concatenation operator},label={lst:concatdefintion}]
[] ++ xs = xs
(x:xs) ++ ys = x:(xs++ys)
\end{lstlisting}

\begin{description}
\item[Base case]
We have to show that property \ref{eq:lengthprop} holds for the base case. The base case, in this example, are the arguments \verb|[]| and an arbitrary list for \verb|ys|. It isn't necessary to replace \verb|ys| because the definition of \verb|++| uses recursion over \verb|xs|. \verb|ys| will always be the same list.
In order to check if property \ref{eq:lengthprop} holds for the base case, we replace \verb|xs| with  \verb|[]|, leading to the following Haskell expression.

\begin{verbatim}
length ([] ++ ys) == length [] + length ys
\end{verbatim}

We will evaluate the expression stepwise on the left-hand side and the right and side separately.
The left-hand side evaluates to

\begin{verbatim}
length ([] ++ ys) -- apply ++
= length ys
\end{verbatim}

The right-hand side evaluates to 

\begin{verbatim}
length [] + length ys     -- apply length []
= 0 + length ys
= length ys
\end{verbatim}

When we evaluate each side of the equation \ref{eq:lengthprop} with the value \verb|[]| for \verb|xs|, the result is \verb|length ys| on both sides. Hence, property \ref{eq:lengthprop} holds for the base case.

\item[Induction step]
We have to show that if equation \ref{eq:lengthprop} holds for any list \verb|xs| then it also holds for \verb|x:xs| (\verb|x:xs| is the list \verb|xs| with a  element \verb|x| attached to its head). Therefore we have to show that the two expressions on both sides of the \verb|=| sign of the  expression in listing \ref{lst:inductionstep} are equal with the assumption that the property of equation \ref{eq:lengthprop} holds (induction hypothesis).
\begin{lstlisting}[caption={Equality expression for induction step},label={lst:inductionstep}]
length ((x:xs) ++ ys) = length (x:xs) + (length ys)
\end{lstlisting}
Again, we evaluate the the left-hand side of equation in listing \ref{lst:inductionstep} step by step.
\begin{verbatim}
length ((x:xs) ++ ys)        -- apply definition of ++
= length (x:(xs ++ ys))      -- apply definition of length
= 1 + length (xs ++ ys)      -- use induction hypothesis
= 1 + length xs + length ys
\end{verbatim}
If we evaluate the right-hand side of the equation in listing \ref{lst:inductionstep} we get:
\begin{verbatim}
length (x:xs) + length ys    -- apply definition of length
1 + length xs + length ys
\end{verbatim}

The last listing shows, that the equality in listing \ref{lst:inductionstep} follows from the induction hypothesis in equation \ref{eq:lengthprop}. This completes the induction step and therefore the proof itself.
\end{description}

The previous example used \glspl{function-definition} of \verb|length| and \verb|(++)|. In order to apply equational reasoning we have to know the function definitions of involved functions or we can rely on already proven properties. 
For example all types of the standard library, that are an instance of a type class, satisfy the type class laws (see \ref{sec:typeclasses}) \cite{yorgey}. Some libraries exhibit properties in their documentation (e.g. pipes library \cite{gonzales13}). 
\section{Example proof for Monoid Laws}
\label{sec:example}
In this example we use equational reasoning (see section \ref{sec:equationalreasoning}) to prove the monoid laws (see section \ref{sec:monoid}) for a new type.
The following example is taken from a blogpost of Gabriel Gonzales \cite{gonzales14}. 

Suppose we want to build a plugin system. A plugin in our example is a \verb|IO| action, that takes a \verb|Char| value and does some work with it (e.g. log to a file, potentially with side effect). We demand the following requirements:
\begin{itemize}
\item We want to be able to add an arbritrary number of plugins.
\item The plugins should be composable.
\item The order we add the plugins must not matter.
\end{itemize}

A monoid instance satisfies all listed requirements. We could use the \verb|mappend| operator to compose several plugins. Hence, if we can prove that the plugins satisfy the monoid laws, we can combine them, using the monoid function \verb|mappend| and we are able to add plugins without concerning about the order of evaluation. In addition they are easier to use because, they will behave as expected.

The following listing shows the use of a plugin. The \verb|logto| function is a plugin. The program will read a character \verb|c| from the command line and apply the given plugin to \verb|c|.

\begin{verbatim}
main = do
    handleChar <- logto
    c <- getChar
    handleChar c
\end{verbatim}

To append additional plugins, e.g. \verb|print2stdout|, we compose a new monoid with the \verb|mappend| function.
\begin{verbatim}
handleChar <- mappend logto print2stdout
\end{verbatim}
The definition of \verb|logTo| is shown in the following listing: 

\begin{verbatim}
logTo :: IO ( Char -> IO ())
logTo = do
    handle <- openFile "log.txt" WriteMode
    return (hPutChar handle)
\end{verbatim}

The plugins are of type \verb|IO ( Char -> IO ())|. Hence we have to provide a monoid instance implementation for the type \verb|IO ( Char -> IO ())|. Instead of writing a specialized instance for \verb|IO ( Char -> IO ())|, we use the general implementation of section \ref{sec:monoid}.
The instance implementation is repeated for convinience:

\begin{lstlisting}[caption={Monoid instance},label={lst:monoidinstance}]
instance (Applicative f, Monoid b) => Monoid (f b) where
    mempty = pure mempty

    mappend = liftA2 mappend
\end{lstlisting}
The generalization has the advantage, that we have to prove the type class law only once for all types that match the general type declaration. The process of verifing a program is cumbersome and time consuming generalization of proofs is desirable.

We will first describe, why the instance implementation of listing \ref{lst:monoidinstance} makes the type \verb|IO ( Char -> IO ())| part of the \verb|Monoid| type class. Then we prove that the monoid laws from section \ref{sec:monoid} hold for the definition in listing \ref{lst:monoidinstance}. We will assume that the instance implementation of listing \ref{lst:monoidinstance} is a monoid in the first part.

\subsection{Generalization of the plugin type }
\label{sec:generalization}

In order to use the instance implementatoin of listing \ref{lst:monoidinstance} the type \verb|IO| has to be part of the type class \verb|Applicative| and \verb|( Char -> IO ())| has to be a \verb|Monoid|. We can use the same argument to check if \verb|( Char -> IO ())|  is a monoid. \verb|( Char -> IO ())| is a monoid if the type \verb|(->) Char| (that's the type of a haskell function) is an applicative and the type \verb|IO ()| is a monoid. We repeat the same argument for \verb|IO ()|.

Here is an overview of all the steps of the argumentation:
\begin{enumerate}
\item show that  \verb|IO ( Char -> IO ())| is of type 
\begin{verbatim}
(Applicative f, Monoid b) => Monoid (f b)
\end{verbatim}
\item show that \verb|Char -> IO ()| is a monoid
\item show that \verb|IO ()| is a monoid
\item show that \verb|()| is a monoid
\end{enumerate}

We show the required properties in reversed order.

\begin{etaremune}
\item The standard library provides a monoid instance for \verb|()| \cite{monoid}. In this article we will believe the farytale of abstraction and assume that the implemention of the standard library satisfies the monoid laws.
\item \verb|IO ()| is a monoid, because \verb|IO| is part of the \verb|Applicative| type class \cite{control.applicative} and \verb|()| is a monoid if the implementation from listing \ref{lst:monoidinstance} satisfies the monoid laws.
\item The type \verb|(->) r| (that's the type of haskell functions) is a part of the \verb|Applicative| type class \cite{control.applicative}.
 \verb|Char -> IO ()| is a monoid because \verb|(->) Char| is a applicative and \verb|IO ()| is a monoid if the implementation from listing \ref{lst:monoidinstance} satisfies the monoid laws.
\item The compiler will use the instance implementation for \verb|Monoid| type class from listing \ref{lst:monoidinstance} for the type \verb|IO (Char -> IO ())| because it matches the type
\begin{verbatim}
(Applicative f, Monoid b) => Monoid (f b)
\end{verbatim}

\end{etaremune} 

Notice that we rely heavily on the assumtion that listing \ref{lst:monoidinstance} satisfies the monoid laws. The next section will prove that the implementation is correct.

\subsection{Proof}
\label{sec:exampleproof}

In this section we will show that the implementation in listing \ref{lst:monoidinstance} satisfies the left identity law of the \verb|Monoid| type class (see section \ref{sec:monoid}.
The left identity law demands that:
\begin{verbatim}
mappend mempty x = x
\end{verbatim}

We will use equational reasoning (see section \ref{sec:equationalreasoning}) to show that the left-hand side is equal to \verb|x|. First we will use the definitions of \verb|mappend| and \verb|mempty| of listing \ref{sec:equationalreasoning} to substitute the left-hand side. Furthermore we lookup the definition of \verb|liftA2| in the source code \cite{control.applicative} to evaluate the expression. 
\verb|liftA2| is defined as follows
\begin{verbatim}
liftA2 f x y = (pure f <*> x) <*> y
\end{verbatim}

\begin{verbatim}
mappend mempty x        
= liftA2 mappend mempty x
= liftA2 mappend (pure mempty) x
= (pure mappend <*> pure mempty) <*> x
\end{verbatim}

One law of the \verb|Applicative| type class says (see section \ref{sec:applicatives}):
\begin{verbatim}
pure f <*> pure x = pure (f x)
\end{verbatim}
We can use this property substitute the left-hand side. Next we write \verb|mappend mempty| as lambda function \verb|\a -> mappend mempty a| and use the monoid law
\begin{verbatim}
mappend mempty x = x
\end{verbatim}
to simplify the expression. In the last step we use the first applicative law ((see section \ref{sec:applicatives}) to rewrite the expression as \verb|x|.
\begin{verbatim}
(pure mappend <*> pure mempty) <*> x
= pure (mappend mempty) <*> x
= pure (\a -> mappend mempty a) <*> x
= pure (\a -> a) <*> x
= pure id <*> x
x
\end{verbatim}

That completes the proof.

The example demonstrated several ideas:
\begin{itemize}
\item Type classes allow us to generalize definitions. A prove for the generalization is valid for all specializations.
\item To prove type class law we can use equational reasoning.c
\item Type class laws (or properties) allow to prove further properties.
\end{itemize}



\section{Pipes: A real world example}
\label{sec:pipes}

Pipes is a streaming library for Haskell. It was build with the following requirements \cite{gonzales13}.
\begin{description}
\item[Effects] Streams has to be effectful
\item[Streaming] Processing in constant memory
\item[Composability] Modules have to be composable
\end{description}

Pipes uses the type class \verb|Category| too keep the API simple and easy to use.

This article will descipe a part of the library with an example.

First we need a input stream, that emits values. Input streams are have the type 
\verb|Producer a m ()|.
Listing \ref{lst:simpleproducer} show a simple Producer. It emits the integers 1,2 and 3. It is also possible to create producers for effectful streams. We use the simple producer from Listing \ref{lst:simpleproducer} for simplicity reasons.

\begin{program}
\begin{verbatim}
produceints123 :: Producer Int IO ()
produceints123 = each [1,2,3]
\end{verbatim}
\caption{Simple Producer}
\label{lst:simpleproducer}
\end{program}

Pipes provide the function \verb|for| to consume a producer. 
\verb|for| \verb|producer| \verb|body| loops over the \verb|producer| and applies a the transformation defined in \verb|body| to every element yielded by the producer. If the body is of type \verb|Effect| it return an \verb|Effect|. If body is of type \verb|Producer| a \verb|Producer| is returned. The type declaration of \verb|for| 
\begin{program}
\begin{verbatim}
for::Monad m=>
Producer b m r
-> (b -> Producer c m ())
-> Producer c m r
\end{verbatim}
\caption{type of for}
\end{program}

A value for the body, could be function, that takes \verb|Int| and returns producer \verb|Producer a IO ()|. Hence the type declarion is \verb|Int -> Producer a IO ()|. We implement a body, that prints the integers to the standard input (strictly speaking this value is an effect of type \verb|Effect Int ()|. But \verb|Effect Int ()| is of type \verb|Producer Int IO () |).

\begin{program}
\begin{verbatim}
print2stdout:: Int -> Effect IO ()
print2stdout x = (lift . putStrLn . show ) x
\end{verbatim}
\caption{Definition of an effect, that prints to stdout}
\label{lst:stdouteffect}
\end{program}

With the producer and the body we can define a \verb|Effect IO ()|
\begin{verbatim}
effect :: Effect IO ()
effect = for produceints123 print2stdout
\end{verbatim}

Because the \verb|body| and of \verb|for| can be of type \verb|a -> Producer|, and the type of the return value is \verb|Producer|, it's possible to interleave several Producers.

If we want to duplicate all element we write another body
\begin{verbatim}
duplicate :: Int -> Producer Int IO ()
duplicate x = yield x >> yield x 
\end{verbatim}

We use \verb|for| to create a another producer. \verb|composed_producer| is a composition of \verb|produceints123| and \verb|duplicate|.
\begin{verbatim}
composed_producer :: Producer Int IO ()
composed_producer = for produceints123 duplicate
\end{verbatim}

To compose \verb|composed_producer| with our \verb|print2stdout| effect, we apply the \verb|for| function

\begin{verbatim}
effect2 :: Effect IO ()
effect2 = for composed_producer print2stdout
\end{verbatim}

In order to compose producers, the pipe library provides the \verb|~>| function. \verb|~>| is defined as follows:
\begin{verbatim}
(f ~> g) x = for (f x) g
\end{verbatim}

Hence we can write compose a body for \verb|for| like this:
\begin{verbatim}
composed_with_yield :: Int -> Effect IO ()
composed_with_yield = duplicate ~> print2stdout
\end{verbatim}

Gabriel Gonzales proved that the \verb|~>| operator is associativ. It has the associativity property.

\begin{verbatim}
 -- Associativity
 (f ~> g) ~> h = f ~> (g ~> h)
\end{verbatim}

Because the \verb|~>| operator is associativ, it doesnt matter in witch order the body is composed. The expected behavior is defined in term of cateory laws. It's possible to prove that the specified laws hold.

\section{Function composition}

\label{sec:functioncomposition}

Function composition in mathematics is defined as

\begin{equation}
  \label{eq:functioncomposition}
  (f \circ g)(x) = f(g(x))
\end{equation}

Haskell functions can be compose with the \verb|.| function.

\begin{figure}
  \centering
\begin{verbatim}
(.) :: (b -> c) -> (a -> b) -> a -> c
f . g = \x -> f (g x)
\end{verbatim}
  \caption{. function}
  \label{fig:compositionfunction}
\end{figure}

Function composition allows us to create a function by composing it with many small functions. 

Functions in Haskell form a category in category theory. They satisfy the category laws.
\begin{description}
\item[Associativity law] 
\item[Left/Right Identity law] 
\end{description}

\bibliographystyle{plain}
\bibliography{a}
\end{document}