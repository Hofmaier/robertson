
\section{Appendix}
\label{sec:appendix}

\subsection{Functor}
\label{sec:functor}

Functor is a type class for types, which can be mapped over. Another way to describe functors is, that they represent some sort of computational context \cite{yorgey}. The general concept of a functor is more abstract and harder to grasp than the concepts of type classes described above \footnote{this is personal impression}.
The most accurate way to describe the type class \verb|Functor| is to give it's declaration \cite{data.functor}:

\begin{verbatim}
class Functor f where
    fmap :: (a -> b) -> f a -> f b
\end{verbatim}

The \verb|f| in the declaration is a type class constructor. Only type constructor can implement \verb|Functor| (\verb|Maybe|, \verb|[]|).

\verb|fmap| takes any function \verb|a -> b| and a value of type \verb|f a| (\verb|f| is the container or context, \verb|a| is the type wrapped inside the functor) and returns a value of type \verb|f b|. 
If \verb|f a| is of type \verb|Maybe| \verb|Int| and the function of type \verb|Int -> String|, \verb|fmap| returns \verb|Maybe String|. 

Instances of \verb|Functor| are:

\begin{description}
\item[List] \verb|map| for lists for is the same as \verb|fmap|.
\item[Either] \verb|Either e a| is a container. \verb|fmap| applies a function to \verb|a|.
\end{description}

To make a type an instance of \verb|Functor|, it has to define \verb|fmap|. In addition the instances of are expected to exhibit certain kinds of properties. The declaration of the type class doesn't reveal this properties. There are described in the type class documentation \cite{data.functor} \cite{Marlow_2010}. This properties are called the functor laws.
The Haskell Compiler doesn't detect violations of the expected laws. All Functor instances in the standard library obey these laws \cite{yorgey} \cite{Lipovaca}.

A Functor instance has to satisfy the following laws.

\begin{description}
\item[Law 1] Mapping the identity function over a functor value, will not change the functor value. Formally
\begin{verbatim}
fmap id  ==  id
\end{verbatim}
\item[Law 2] It doesn't matter if we compose two functions and them map them over a functor or if we first map one function over the functor and then map the other function. Formally
\begin{verbatim}
fmap (g . h) = fmap g . fmap h
\end{verbatim}
This is the same as \verb|fmap (g . h) = fmap g (fmap h)|
\end{description}

If we can prove that a type satisfies these laws, we can make assumptions about how the the type will act. We know that \verb|fmap| will not change the structure or the context of the functor.
And we know that \verb|fmap| only maps the function over the functor and nothing else. 

If we know, that a type satisfy the laws, we are able to deduce further properties for our own types. In section \ref{sec:example} give an example of this process.

\subsection{Applicative Functor}
\label{sec:applicatives}

Applicative functors are an abstract characterization of an applicative style of effectful programming \cite{mcbride} \cite{control.applicative}

The \verb|Applicative| type class encapsulates the following idea. What if you have a function wrapped in a \verb|Functor| (e.g. \verb|Maybe (Int -> Int -> Int)|) and you want to apply the function to another functor (e.g. \verb|Maybe Int|). For example we want to map \verb|Just (3 *)|, a function encapsulated inside a functor, over \verb|Just 23|, another functor with an encapsulated \verb|Int|? \verb|fmap| doesn't work here, because it expects a normal function \verb|a -> b| as first parameter. That's where the \verb|Applicative| type class come in. 
The \verb|Applicative| type class is defined as follows \cite{control.applicative}.
\begin{verbatim}
class (Functor f) => Applicative f where
    pure :: a -> f a
    (<*>) :: f (a -> b) -> f a -> f b
\end{verbatim}

The type class declaration demands the type \verb|f| to be a functor. Every type that is part of \verb|Applicative| is part of \verb|Functor|. Hence we can use \verb|fmap| with \verb|Applicative| instances.
Applicatives are enhanced functors. In addition to to \verb|fmap| we can use the \verb|<*>| operator to chain several applicative values together.

The \verb|pure| function puts a value of type \verb|a| in a default context. If applied with a function \verb|a -> b| pure return a functor with a function inside, hence the first argument to \verb|<*>|.

It possible to chain applicative together as follows:
\begin{verbatim}
ghci> pure (*) <*> Just 3 <*> Just 23
Just 69
\end{verbatim}

The following type are instances of the type class \verb|Applicative|:

\begin{itemize}
\item Maybe
\item Lists
\item IO
\end{itemize}

There are several laws that \verb|Applicative| instances should satisfy \cite{mcbride} \cite{control.applicative}. This article will use only the following :

\begin{enumerate}
\item \verb|pure id <*> v = v|
\item \verb|fmap g x = pure g <*> x|
\item \verb|pure f <*> pure x = pure (f x)|
\end{enumerate}

The second law says that applying \verb|fmap| over a normal function \verb|g| over a functor \verb|x| is the same as putting \verb|g| in a default context and mapping the resulting function over \verb|x|.

The third law says that it doesn't matter if we put the values \verb|f| and \verb|x| in a default context first and then apply \verb|<*>| or if we call \verb|f| on \verb|x| first and then put them in a default context. We will use this property to prove another property in section \ref{sec:exampleproof}

\subsection{Monoid}
\label{sec:monoiddefinition}

The type class \verb|Monoid| is defined in the Module \verb|Data.Monoid|. Listing \ref{lst:monoiddefinition} shows the definition.
\begin{lstlisting}[label={lst:monoiddefinition}, caption={Definition of type class monoid}]
class Monoid m where
  mempty :: m
  mappend :: m -> m -> m
  mconcat :: [m] -> m
  mconcat = foldr mappend mempty
\end{lstlisting}